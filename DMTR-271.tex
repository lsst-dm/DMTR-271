\documentclass[DM,lsstdraft,STR,toc]{lsstdoc}
\usepackage{geometry}
\usepackage{longtable,booktabs}
\usepackage{enumitem}
\usepackage{arydshln}

\input meta.tex

\providecommand{\tightlist}{
  \setlength{\itemsep}{0pt}\setlength{\parskip}{0pt}}

\setcounter{tocdepth}{4}

\begin{document}

\def\milestoneName{Ldm-Gen3: Gen 3 Butler Acceptance Testing}
\def\milestoneId{LVV-P77}
\def\product{Software Products}

\setDocCompact{true}

\title{ LVV-P77 Ldm-Gen3: Gen 3 Butler Acceptance Testing Test Plan and Report}
\setDocRef{\lsstDocType-\lsstDocNum}
\date{\vcsdate}
\setDocUpstreamLocation{\url{https://github.com/lsst/lsst-texmf/examples}}
\author{ Robert Gruendl }

\input history_and_info.tex


\setDocAbstract{
This is the test plan and report for LVV-P77 (Ldm-Gen3: Gen 3 Butler Acceptance Testing),
an LSST level 2 milestone pertaining to the Data Management Subsystem.
}


\maketitle

\section{Introduction}
\label{sect:intro}


\subsection{Objectives}
\label{sect:objectives}

 The goal of this test is to demonstrate that the Gen3 Butler software
project has sufficiently matured that subsequent DM development should
begin focusing on adoption of Gen3 Butler software are repositories
throughout the DM software project (i.e. that deprecation of Gen2 Butler
usage within the project can begin).



\subsection{System Overview}
\label{sect:systemoverview}

 The Gen3 refactoring of the Butler is central to evolution of the
overall DM software design and has repercussions throughout the rest of
the DM project. ~ This test plan is designed to verify that minimal
requirements have been met and the DM project can now begin the process
of integrating the Gen3 Butler within the pipelines and analysis tools.
~ Those minimal requirements are that:

\begin{enumerate}
\tightlist
\item
  possible to ingest all raw dataset types currently hosted at NCSA (for
  supported instruments)
\item
  cp\_pipe equivalent under Gen3 is available
\item
  developers can run a pipeline with a single-node using pipetask
\item
  an RC2-like processing is possible
\item
  a 3-tract RC2 test run can be performed in a reasonable time (under a
  week)
\item
  Calibration Product Pipelines (CPP) can be run at scales to support
  above investigations
\item
  Batch Processing System (BPS) is available to support testing at
  larger scales
\end{enumerate}

In addition, at the time these tests occur the Gen3 Butler schema be
considered stable enough that changes no longer occur on a weekly basis
(i.e force re-ingestion/migration of existing repositories are no longer
a weekly occurrence). Changes requiring wholesale reingestion/migration
may still be required but will occur in a regimented
manner.\\[2\baselineskip]\textbf{Applicable Documents:}\\
\citeds{LDM-592}: Data Access Use Cases\\
\citeds{LDM-556}: Data Management Middleware Requirements\\
\citeds{LDM-639}: Data Management Acceptance Test Specification\\[2\baselineskip]


\subsection{Document Overview}
\label{sect:docoverview}

This document was generated from Jira, obtaining the relevant information from the 
\href{https://jira.lsstcorp.org/secure/Tests.jspa#/testPlan/LVV-P77}{LVV-P77}
~Jira Test Plan and related Test Cycles (
  \href{https://jira.lsstcorp.org/secure/Tests.jspa#/testCycle/LVV-C160}{LVV-C160}
  \href{https://jira.lsstcorp.org/secure/Tests.jspa#/testCycle/LVV-C162}{LVV-C162}
).

Section \ref{sect:intro} provides an overview of the test campaign, the system under test (\product{}),
the applicable documentation, and explains how this document is organized.
Section \ref{sect:configuration}  describes the configuration used for this test.
Section \ref{sect:personnel} describes the necessary roles and lists the individuals assigned to them.
%Section \ref{sect:plannedtestactivities} provides the list of planned test cycles and test cases,
including all relevant information that fully describes the test campaign.

Section \ref{sect:overview} provides a summary of the test results, including an overview in Table \ref{table:summary},
an overall assessment statement and suggestions for possible improvements.
Section \ref{sect:detailedtestresults} provides detailed results for each step in each test case.

The current status of test plan LVV-P77 in Jira is \textbf{ Draft }.

\subsection{References}
\label{sect:references}
\renewcommand{\refname}{}
\bibliography{lsst,refs,books,refs_ads}
\section{Test Configuration}
\label{sect:configuration}

\subsection{Data Collection}

  Observing is not required for this test campaign.

\subsection{Verification Environment}
\label{sect:hwconf}
  These tests assume a stable weekly stack which supports Gen3 running of
the above, that services that automatically ingest new data can support
on-going ingestion to Gen3 repositories (i.e. DBB shared spaces and OODS
support serving data through Gen3), and that batch processing services
can support pipeline execution of Gen3 products.




\newpage
\section{Personnel}
\label{sect:personnel}

The personnel involved in the test campaign are shown in the following table.

\begin{longtable}{p{3cm}p{3cm}p{3cm}p{6cm}}
\hline
\multicolumn{2}{r}{Test Plan (LVV-P77) owner:} &
\multicolumn{2}{l}{\textbf{ Robert Gruendl } }\\\hline
\multicolumn{2}{r}{ LVV-C160 owner:} &
\multicolumn{2}{l}{\textbf{
    Robert Gruendl
}
} \\\hline
\textbf{Test Case} & \textbf{Assigned to} & \textbf{Executed by} & \textbf{Additional Test Personnel} \\ \hline
\href{https://jira.lsstcorp.org/secure/Tests.jspa#/testCase/LVV-T1984}{LVV-T1984}
& {\small Leanne Guy } & {\small  } &
\begin{minipage}[]{6cm}
\smallskip
{\small  }
\medskip
\end{minipage}
\\ \hline
\href{https://jira.lsstcorp.org/secure/Tests.jspa#/testCase/LVV-T1982}{LVV-T1982}
& {\small Leanne Guy } & {\small  } &
\begin{minipage}[]{6cm}
\smallskip
{\small  }
\medskip
\end{minipage}
\\ \hline
\href{https://jira.lsstcorp.org/secure/Tests.jspa#/testCase/LVV-T1987}{LVV-T1987}
& {\small Leanne Guy } & {\small  } &
\begin{minipage}[]{6cm}
\smallskip
{\small  }
\medskip
\end{minipage}
\\ \hline
\href{https://jira.lsstcorp.org/secure/Tests.jspa#/testCase/LVV-T1983}{LVV-T1983}
& {\small Leanne Guy } & {\small  } &
\begin{minipage}[]{6cm}
\smallskip
{\small  }
\medskip
\end{minipage}
\\ \hline
\multicolumn{2}{r}{ LVV-C162 owner:} &
\multicolumn{2}{l}{\textbf{
    Undefined
}
} \\\hline
\textbf{Test Case} & \textbf{Assigned to} & \textbf{Executed by} & \textbf{Additional Test Personnel} \\ \hline
\href{https://jira.lsstcorp.org/secure/Tests.jspa#/testCase/LVV-T1985}{LVV-T1985}
& {\small Leanne Guy } & {\small  } &
\begin{minipage}[]{6cm}
\smallskip
{\small  }
\medskip
\end{minipage}
\\ \hline
\end{longtable}

\newpage

\section{Test Campaign Overview}
\label{sect:overview}

\subsection{Summary}
\label{sect:summarytable}

\begin{longtable}{p{2cm}p{2.5cm}p{9cm}p{2.5cm}}
\toprule
\multicolumn{3}{p{13.5cm}}{ Test Plan {\bf LVV-P77: LDM-GEN3: Gen 3 Butler Acceptance Testing }} & Draft \\\hline

  \multicolumn{3}{p{13.5cm}}{ Test Cycle {\bf LVV-C160: LDM-503-GEN3-01: Gen 3 Butler Acceptance Testing }} & Not Executed \\\hline

  {\bf \footnotesize test case} & {\bf \footnotesize status} & {\bf \footnotesize comment} & {\bf \footnotesize issues} \\\toprule

\href{https://jira.lsstcorp.org/secure/Tests.jspa#/testCase/LVV-T1984}{LVV-T1984}
    & Not Executed &
    \begin{minipage}[]{9cm}
    \smallskip
    
    \medskip
    \end{minipage}
    &
    \\\hline
\href{https://jira.lsstcorp.org/secure/Tests.jspa#/testCase/LVV-T1982}{LVV-T1982}
    & Not Executed &
    \begin{minipage}[]{9cm}
    \smallskip
    
    \medskip
    \end{minipage}
    &
    \\\hline
\href{https://jira.lsstcorp.org/secure/Tests.jspa#/testCase/LVV-T1987}{LVV-T1987}
    & Not Executed &
    \begin{minipage}[]{9cm}
    \smallskip
    
    \medskip
    \end{minipage}
    &
    \\\hline
\href{https://jira.lsstcorp.org/secure/Tests.jspa#/testCase/LVV-T1983}{LVV-T1983}
    & Not Executed &
    \begin{minipage}[]{9cm}
    \smallskip
    
    \medskip
    \end{minipage}
    &
    \\\hline

  \multicolumn{3}{p{13.5cm}}{ Test Cycle {\bf LVV-C162: LDM-503-GEN3-01: Gen 3 Ingest raw dataset }} & Not Executed \\\hline

  {\bf \footnotesize test case} & {\bf \footnotesize status} & {\bf \footnotesize comment} & {\bf \footnotesize issues} \\\toprule

\href{https://jira.lsstcorp.org/secure/Tests.jspa#/testCase/LVV-T1985}{LVV-T1985}
    & Not Executed &
    \begin{minipage}[]{9cm}
    \smallskip
    
    \medskip
    \end{minipage}
    &
    \\\hline
\caption{Test Campaign Summary}
\label{table:summary}
\end{longtable}

\subsection{Overall Assessment}
\label{sect:overallassessment}

Not yet available.

\subsection{Recommended Improvements}
\label{sect:recommendations}

Not yet available.

\newpage
\section{Detailed Test Results}
\label{sect:detailedtestresults}

\subsection{Test Cycle LVV-C160 }

Open test cycle {\it \href{https://jira.lsstcorp.org/secure/Tests.jspa#/testrun/LVV-C160}{LDM-503-GEN3-01: Gen 3 Butler Acceptance Testing}} in Jira.

LDM-503-GEN3-01: Gen 3 Butler Acceptance Testing\\
Status: Not Executed

This test cycle is meant to demonstrate that the Gen3 butler and
associated database and pipeline interfaces have matured to the point
where they can replace the Gen2 butler. ~The test cases outlined here:

\begin{enumerate}
\tightlist
\item
  use a series of modest pipeline executions to show that the Gen3
  software can support all future pipeline development, ~
\item
  those pipeline executions also show that a batch processing system
  (BPS) is available to enable that processing, and
\item
  demonstrate through inspection that documentation for developers
  exists.
\end{enumerate}

{[}GCM{]}\\
I see here 2 main groups of test:\\
i- ingestion of different raw dataset:\\
in case we have one single test case for ingestion of data in Gen3
(T1985), we need to have one test cycle for each type of raw data
ingested\\
ii- various pipeline processing\\
all these test cases can be part of a single test cycle. In case T1985
is divided in many test cases, one per dataset, those tests can also be
included here.\\
\hspace*{0.333em} ~ ~ ~These test cases seem to be generic Science
Pipelines test cases, ae we sure they have not already been defined and
executed?

\subsubsection{Software Version/Baseline}
Not provided.

\subsubsection{Configuration}
Gen3 Butler repositories with test data are available within DBB spaces.
~Weekly DM stack has Gen3 and BPS elements present for tests.

\subsubsection{Test Cases in LVV-C160 Test Cycle}

\paragraph{Test Case LVV-T1984 - Demonstrate documentation/examples of Gen3 usage and cp\_pipe
equivalent. }\mbox{}\\

Open  \href{https://jira.lsstcorp.org/secure/Tests.jspa#/testCase/LVV-T1984}{\textit{ LVV-T1984 } }
test case in Jira.

Demonstrate the existence of fundamental documentation necessary to aid
Gen2 users with the transition to Gen3 use.

\textbf{ Preconditions}:\\


Execution status: {\bf Not Executed }

Final comment:\\


Detailed steps results:

\begin{longtable}{p{1cm}p{15cm}}
\hline
{Step} & Step Details\\ \hline
1 & Description \\
 & \begin{minipage}[t]{15cm}
{\footnotesize

\medskip }
\end{minipage}
\\ \cdashline{2-2}


 & Expected Result \\
 & \begin{minipage}[t]{15cm}{\footnotesize

\medskip }
\end{minipage} \\ \cdashline{2-2}

 & Actual Result \\
 & \begin{minipage}[t]{15cm}{\footnotesize

\medskip }
\end{minipage} \\ \cdashline{2-2}

 & Status: \textbf{ Not Executed } \\ \hline

\end{longtable}

\paragraph{Test Case LVV-T1982 - Run a pipeline on a single node using pipetask. }\mbox{}\\

Open  \href{https://jira.lsstcorp.org/secure/Tests.jspa#/testCase/LVV-T1982}{\textit{ LVV-T1982 } }
test case in Jira.

To show that individual users have the ability to run either locally (w/
sqlite) or generally (w/ Postgres) using Gen3 Butler infrastructure.

\textbf{ Preconditions}:\\
This test requires that Gen3 Butler infrastructure and underlying pipets
have been integrated. ~It further requires (in spirit) that gen3 schema
stability has been reached to facilitate comparison of pipeline results
with further stack development can be compared.

Execution status: {\bf Not Executed }

Final comment:\\


Detailed steps results:

\begin{longtable}{p{1cm}p{15cm}}
\hline
{Step} & Step Details\\ \hline
1 & Description \\
 & \begin{minipage}[t]{15cm}
{\footnotesize

\medskip }
\end{minipage}
\\ \cdashline{2-2}


 & Expected Result \\
 & \begin{minipage}[t]{15cm}{\footnotesize

\medskip }
\end{minipage} \\ \cdashline{2-2}

 & Actual Result \\
 & \begin{minipage}[t]{15cm}{\footnotesize

\medskip }
\end{minipage} \\ \cdashline{2-2}

 & Status: \textbf{ Not Executed } \\ \hline

\end{longtable}

\paragraph{Test Case LVV-T1987 - Run Calibration Products Processing (CPP) }\mbox{}\\

Open  \href{https://jira.lsstcorp.org/secure/Tests.jspa#/testCase/LVV-T1987}{\textit{ LVV-T1987 } }
test case in Jira.

Demonstrate that basic calibration processing from Gen2 era has been
enabled within Gen3 environment. ~ This test is not concerned with large
scales but merely demonstrates that Gen3 capability to generate
calibration products (i.e. they are no longer required to be generated
in Gen2 and then migrated to Gen3).

\textbf{ Preconditions}:\\


Execution status: {\bf Not Executed }

Final comment:\\


Detailed steps results:

\begin{longtable}{p{1cm}p{15cm}}
\hline
{Step} & Step Details\\ \hline
1 & Description \\
 & \begin{minipage}[t]{15cm}
{\footnotesize

\medskip }
\end{minipage}
\\ \cdashline{2-2}


 & Expected Result \\
 & \begin{minipage}[t]{15cm}{\footnotesize

\medskip }
\end{minipage} \\ \cdashline{2-2}

 & Actual Result \\
 & \begin{minipage}[t]{15cm}{\footnotesize

\medskip }
\end{minipage} \\ \cdashline{2-2}

 & Status: \textbf{ Not Executed } \\ \hline

\end{longtable}

\paragraph{Test Case LVV-T1983 - Mini RC2 processing capability }\mbox{}\\

Open  \href{https://jira.lsstcorp.org/secure/Tests.jspa#/testCase/LVV-T1983}{\textit{ LVV-T1983 } }
test case in Jira.

Demonstrate that a typical 3-tract RC2 data processing is possible using
the Gen3 system and the nascent Batch Production Service (BPS). ~This
test is meant to demonstrate that Gen3 + BPS systems are capable of
supporting future DM development by demonstrating that processing
routinely used by developers for benchmarking/testing improvements can
be performed in a reasonable time. ~

\textbf{ Preconditions}:\\


Execution status: {\bf Not Executed }

Final comment:\\


Detailed steps results:

\begin{longtable}{p{1cm}p{15cm}}
\hline
{Step} & Step Details\\ \hline
1 & Description \\
 & \begin{minipage}[t]{15cm}
{\footnotesize

\medskip }
\end{minipage}
\\ \cdashline{2-2}


 & Expected Result \\
 & \begin{minipage}[t]{15cm}{\footnotesize

\medskip }
\end{minipage} \\ \cdashline{2-2}

 & Actual Result \\
 & \begin{minipage}[t]{15cm}{\footnotesize

\medskip }
\end{minipage} \\ \cdashline{2-2}

 & Status: \textbf{ Not Executed } \\ \hline

\end{longtable}

\subsection{Test Cycle LVV-C162 }

Open test cycle {\it \href{https://jira.lsstcorp.org/secure/Tests.jspa#/testrun/LVV-C162}{LDM-503-GEN3-01: Gen 3 Ingest raw dataset}} in Jira.

LDM-503-GEN3-01: Gen 3 Ingest raw dataset\\
Status: Not Executed

In the context of the milestone LDM-503-GEN3-01, Gen 3 Butler readiness,
this test cycle is defining the configuration and the dataset for
running a generic \textbf{Raw Data Ingestion Into Gen3 Butler} test
case. ~ There are currently 5 data sources that require verification as
they are the central products that will be produced by Rubin or are used
as precursor sets in the development/verification of the data management
software systems. ~The current raw data products that are deemed central
to DM development and testing are those from AuxTel/LATISS, ComCam, and
~precursor data from HyperSuprimeCam (HSC). ~Note, further tests using
LSSTCam (currently only preliminary BOT data from the SLAC test stand
are available) or precursor sets from the Dark Energy Camera (DECam)
could be added but since these types do not exactly fit the central
model used for LSST they are not tied directly to requirements.

\subsubsection{Software Version/Baseline}
LSST DM Stack with Gen3 Butler.

\subsubsection{Configuration}
Three separate raw data type, those from: AuxTel/LATISS, ComCam, and HSC
(e.g. a CI\_HSC raw) should be ingested when this test is executed.

\subsubsection{Test Cases in LVV-C162 Test Cycle}

\paragraph{Test Case LVV-T1985 - Verify daf\_butler raw data ingest }\mbox{}\\

Open  \href{https://jira.lsstcorp.org/secure/Tests.jspa#/testCase/LVV-T1985}{\textit{ LVV-T1985 } }
test case in Jira.

Demonstrate that a raw data type can be successfully ingested into a
Butler repository.\\[2\baselineskip]There should be the following
steps:\\
- verify that the Gen3 repository is available\\
- ingest the raw data defined in the test cycle (to be divided in
multiple steps in case it makes sense)\\
- verify that the raw data is available in the repository as expected\\
In the case that the second step is different, depending on the raw
dataset, at this point it is necessary to have a different test case for
each of them, and multiple test cycles are not
needed.\\[2\baselineskip]To note that, the original name ``Ingest all
the raw datasets that are currently hosted at NCSA\ldots{}'' looks more
to be a milestone, than a test case definition.

\textbf{ Preconditions}:\\
In order to run this test, a Gen3 daf butler should be deployed and
ready to use, with access to the filesystems where the raw data to
ingest is stored.

Execution status: {\bf Not Executed }

Final comment:\\


Detailed steps results:

\begin{longtable}{p{1cm}p{15cm}}
\hline
{Step} & Step Details\\ \hline
1 & Description \\
 & \begin{minipage}[t]{15cm}
{\footnotesize
Verify that a Butler repository is available for the {HSC RC2}⁠~\\
(Note this either needs to be a test repository or if an existing
repository is used then the raw data should not already be present.)

\medskip }
\end{minipage}
\\ \cdashline{2-2}

 & Test Data \\
 & \begin{minipage}[t]{15cm}{\footnotesize
{HSC RC2}⁠~

\medskip }
\end{minipage} \\ \cdashline{2-2}

 & Expected Result \\
 & \begin{minipage}[t]{15cm}{\footnotesize
Repository or {HSC RC2}⁠ ~is available

\medskip }
\end{minipage} \\ \cdashline{2-2}

 & Actual Result \\
 & \begin{minipage}[t]{15cm}{\footnotesize

\medskip }
\end{minipage} \\ \cdashline{2-2}

 & Status: \textbf{ Not Executed } \\ \hline

2 & Description \\
 & \begin{minipage}[t]{15cm}
{\footnotesize
Verify that a Butler repository is available for the {AuxTel/LATISS}⁠~\\
(Note this either needs to be a test repository or if an existing
repository is used then the raw data should not already be present.)

\medskip }
\end{minipage}
\\ \cdashline{2-2}

 & Test Data \\
 & \begin{minipage}[t]{15cm}{\footnotesize
{AuxTel/LATISS}⁠~

\medskip }
\end{minipage} \\ \cdashline{2-2}

 & Expected Result \\
 & \begin{minipage}[t]{15cm}{\footnotesize
Repository or {AuxTel/LATISS}⁠ ~is available

\medskip }
\end{minipage} \\ \cdashline{2-2}

 & Actual Result \\
 & \begin{minipage}[t]{15cm}{\footnotesize

\medskip }
\end{minipage} \\ \cdashline{2-2}

 & Status: \textbf{ Not Executed } \\ \hline

3 & Description \\
 & \begin{minipage}[t]{15cm}
{\footnotesize
Verify that a Butler repository is available for the {ComCam}⁠~\\
(Note this either needs to be a test repository or if an existing
repository is used then the raw data should not already be present.)

\medskip }
\end{minipage}
\\ \cdashline{2-2}

 & Test Data \\
 & \begin{minipage}[t]{15cm}{\footnotesize
{ComCam}⁠~

\medskip }
\end{minipage} \\ \cdashline{2-2}

 & Expected Result \\
 & \begin{minipage}[t]{15cm}{\footnotesize
Repository or {ComCam}⁠ ~is available

\medskip }
\end{minipage} \\ \cdashline{2-2}

 & Actual Result \\
 & \begin{minipage}[t]{15cm}{\footnotesize

\medskip }
\end{minipage} \\ \cdashline{2-2}

 & Status: \textbf{ Not Executed } \\ \hline

4 & Description \\
 & \begin{minipage}[t]{15cm}
{\footnotesize
Verify that a Butler repository is available for the {DESC DC2}⁠~\\
(Note this either needs to be a test repository or if an existing
repository is used then the raw data should not already be present.)

\medskip }
\end{minipage}
\\ \cdashline{2-2}

 & Test Data \\
 & \begin{minipage}[t]{15cm}{\footnotesize
{DESC DC2}⁠~

\medskip }
\end{minipage} \\ \cdashline{2-2}

 & Expected Result \\
 & \begin{minipage}[t]{15cm}{\footnotesize
Repository or {DESC DC2}⁠ ~is available

\medskip }
\end{minipage} \\ \cdashline{2-2}

 & Actual Result \\
 & \begin{minipage}[t]{15cm}{\footnotesize

\medskip }
\end{minipage} \\ \cdashline{2-2}

 & Status: \textbf{ Not Executed } \\ \hline

5 & Description \\
 & \begin{minipage}[t]{15cm}
{\footnotesize
Ingest {HSC RC2}⁠~raw data into repo

\medskip }
\end{minipage}
\\ \cdashline{2-2}

 & Test Data \\
 & \begin{minipage}[t]{15cm}{\footnotesize
{HSC RC2}⁠~

\medskip }
\end{minipage} \\ \cdashline{2-2}

 & Expected Result \\
 & \begin{minipage}[t]{15cm}{\footnotesize
Tool reports data ingest successful for {HSC RC2}⁠

\medskip }
\end{minipage} \\ \cdashline{2-2}

 & Actual Result \\
 & \begin{minipage}[t]{15cm}{\footnotesize

\medskip }
\end{minipage} \\ \cdashline{2-2}

 & Status: \textbf{ Not Executed } \\ \hline

6 & Description \\
 & \begin{minipage}[t]{15cm}
{\footnotesize
Ingest {AuxTel/LATISS}⁠~raw data into repo

\medskip }
\end{minipage}
\\ \cdashline{2-2}

 & Test Data \\
 & \begin{minipage}[t]{15cm}{\footnotesize
{AuxTel/LATISS}⁠~

\medskip }
\end{minipage} \\ \cdashline{2-2}

 & Expected Result \\
 & \begin{minipage}[t]{15cm}{\footnotesize
Tool reports data ingest successful for {AuxTel/LATISS}⁠

\medskip }
\end{minipage} \\ \cdashline{2-2}

 & Actual Result \\
 & \begin{minipage}[t]{15cm}{\footnotesize

\medskip }
\end{minipage} \\ \cdashline{2-2}

 & Status: \textbf{ Not Executed } \\ \hline

7 & Description \\
 & \begin{minipage}[t]{15cm}
{\footnotesize
Ingest {ComCam}⁠~raw data into repo

\medskip }
\end{minipage}
\\ \cdashline{2-2}

 & Test Data \\
 & \begin{minipage}[t]{15cm}{\footnotesize
{ComCam}⁠~

\medskip }
\end{minipage} \\ \cdashline{2-2}

 & Expected Result \\
 & \begin{minipage}[t]{15cm}{\footnotesize
Tool reports data ingest successful for {ComCam}⁠

\medskip }
\end{minipage} \\ \cdashline{2-2}

 & Actual Result \\
 & \begin{minipage}[t]{15cm}{\footnotesize

\medskip }
\end{minipage} \\ \cdashline{2-2}

 & Status: \textbf{ Not Executed } \\ \hline

8 & Description \\
 & \begin{minipage}[t]{15cm}
{\footnotesize
Ingest {DESC DC2}⁠~raw data into repo

\medskip }
\end{minipage}
\\ \cdashline{2-2}

 & Test Data \\
 & \begin{minipage}[t]{15cm}{\footnotesize
{DESC DC2}⁠~

\medskip }
\end{minipage} \\ \cdashline{2-2}

 & Expected Result \\
 & \begin{minipage}[t]{15cm}{\footnotesize
Tool reports data ingest successful for {DESC DC2}⁠

\medskip }
\end{minipage} \\ \cdashline{2-2}

 & Actual Result \\
 & \begin{minipage}[t]{15cm}{\footnotesize

\medskip }
\end{minipage} \\ \cdashline{2-2}

 & Status: \textbf{ Not Executed } \\ \hline

9 & Description \\
 & \begin{minipage}[t]{15cm}
{\footnotesize
Query repository to verify that ingestion of {HSC RC2}⁠~ occurred.

\medskip }
\end{minipage}
\\ \cdashline{2-2}

 & Test Data \\
 & \begin{minipage}[t]{15cm}{\footnotesize
{HSC RC2}⁠~

\medskip }
\end{minipage} \\ \cdashline{2-2}

 & Expected Result \\
 & \begin{minipage}[t]{15cm}{\footnotesize
{HSC RC2}⁠ data are found by query.~

\medskip }
\end{minipage} \\ \cdashline{2-2}

 & Actual Result \\
 & \begin{minipage}[t]{15cm}{\footnotesize

\medskip }
\end{minipage} \\ \cdashline{2-2}

 & Status: \textbf{ Not Executed } \\ \hline

10 & Description \\
 & \begin{minipage}[t]{15cm}
{\footnotesize
Query repository to verify that ingestion of {AuxTel/LATISS}⁠~ occurred.

\medskip }
\end{minipage}
\\ \cdashline{2-2}

 & Test Data \\
 & \begin{minipage}[t]{15cm}{\footnotesize
{AuxTel/LATISS}⁠~

\medskip }
\end{minipage} \\ \cdashline{2-2}

 & Expected Result \\
 & \begin{minipage}[t]{15cm}{\footnotesize
{AuxTel/LATISS}⁠ data are found by query.~

\medskip }
\end{minipage} \\ \cdashline{2-2}

 & Actual Result \\
 & \begin{minipage}[t]{15cm}{\footnotesize

\medskip }
\end{minipage} \\ \cdashline{2-2}

 & Status: \textbf{ Not Executed } \\ \hline

11 & Description \\
 & \begin{minipage}[t]{15cm}
{\footnotesize
Query repository to verify that ingestion of {ComCam}⁠~ occurred.

\medskip }
\end{minipage}
\\ \cdashline{2-2}

 & Test Data \\
 & \begin{minipage}[t]{15cm}{\footnotesize
{ComCam}⁠~

\medskip }
\end{minipage} \\ \cdashline{2-2}

 & Expected Result \\
 & \begin{minipage}[t]{15cm}{\footnotesize
{ComCam}⁠ data are found by query.~

\medskip }
\end{minipage} \\ \cdashline{2-2}

 & Actual Result \\
 & \begin{minipage}[t]{15cm}{\footnotesize

\medskip }
\end{minipage} \\ \cdashline{2-2}

 & Status: \textbf{ Not Executed } \\ \hline

12 & Description \\
 & \begin{minipage}[t]{15cm}
{\footnotesize
Query repository to verify that ingestion of {DESC DC2}⁠~ occurred.

\medskip }
\end{minipage}
\\ \cdashline{2-2}

 & Test Data \\
 & \begin{minipage}[t]{15cm}{\footnotesize
{DESC DC2}⁠~

\medskip }
\end{minipage} \\ \cdashline{2-2}

 & Expected Result \\
 & \begin{minipage}[t]{15cm}{\footnotesize
{DESC DC2}⁠ data are found by query.~

\medskip }
\end{minipage} \\ \cdashline{2-2}

 & Actual Result \\
 & \begin{minipage}[t]{15cm}{\footnotesize

\medskip }
\end{minipage} \\ \cdashline{2-2}

 & Status: \textbf{ Not Executed } \\ \hline

\end{longtable}



% This appendix is put in as part of the template. You may edit and add to it.
% It is not overwritten by Docsteady.

\newpage
\appendix
\section{Documentation}
The verification process is defined in \citeds{LSE-160}.
The use of Docsteady to format Jira information in various test and planing documents is
described in \citeds{DMTN-140} and practical commands are given in \citeds{DMTN-178}.

\section{Acronyms used in this document}\label{sec:acronyms}
\input{acronyms.tex}

\newpage

% Uncomment this if Docsteady makes you additional appendix
%% generated from JIRA project LVV
% using template at /Users/womullan/LSSTgit/docsteady/docsteady/templates/tpr-appendix.latex.jinja2.
% using docsteady version 2.2.post4+g100285e.d20210519
% Please do not edit -- update information in Jira instead
\section{Traceability}

\begin{longtable}{p{3cm}p{3cm}L{9cm}}
\hline
\textbf{Test Case} & \textbf{VE Key} & \textbf{VE Summary} \\ \hline
\href{https://jira.lsstcorp.org/secure/Tests.jspa#/testCase/LVV-T1982}{LVV-T1982} &
 & \\ \hline
\href{https://jira.lsstcorp.org/secure/Tests.jspa#/testCase/LVV-T1983}{LVV-T1983} &
 & \\ \hline
\href{https://jira.lsstcorp.org/secure/Tests.jspa#/testCase/LVV-T1984}{LVV-T1984} &
 & \\ \hline
\href{https://jira.lsstcorp.org/secure/Tests.jspa#/testCase/LVV-T1985}{LVV-T1985} &
  \href{https://jira.lsstcorp.org/browse/LVV-130}{LVV-130}
  & DMS-REQ-0299-V-01: Data Product Ingest
 \\ \cdashline{2-3}
\hline
\href{https://jira.lsstcorp.org/secure/Tests.jspa#/testCase/LVV-T1987}{LVV-T1987} &
  \href{https://jira.lsstcorp.org/browse/LVV-120}{LVV-120}
  & DMS-REQ-0289-V-01: Calibration Production Processing
 \\ \cdashline{2-3}
\hline
\end{longtable}


\end{document}
