% generated from JIRA project LVV
% using template at /usr/share/miniconda/envs/docsteady-env/lib/python3.7/site-packages/docsteady/templates/tpr.latex.jinja2.
% using docsteady version 2.2.4
% Please do not edit -- update information in Jira instead
\documentclass[dm,lsstdraft,STR,toc]{lsstdoc}
\usepackage{geometry}
\usepackage{longtable,booktabs}
\usepackage{enumitem}
\usepackage{arydshln}
\usepackage{attachfile}
\usepackage{array}
\usepackage{dashrule}

\newcolumntype{L}[1]{>{\raggedright\let\newline\\\arraybackslash\hspace{0pt}}p{#1}}

\input meta.tex

\newcommand{\attachmentsUrl}{https://github.com/\gitorg/\lsstDocType-\lsstDocNum/blob/\gitref/attachments}
\providecommand{\tightlist}{
  \setlength{\itemsep}{0pt}\setlength{\parskip}{0pt}}

\setcounter{tocdepth}{4}

\begin{document}

\def\milestoneName{Gen 3 Butler Acceptance Testing}
\def\milestoneId{LDM-GEN3}
\def\product{Software Products}

\setDocCompact{true}

\title{LDM-GEN3: Gen 3 Butler Acceptance Testing Test Plan and Report}
\setDocRef{\lsstDocType-\lsstDocNum}
\date{ 2022-01-27 }
\author{ Robert Gruendl }

% Most recent last
\setDocChangeRecord{
\addtohist{}{2020-10-30}{First draft}{Robert Gruendl}
\addtohist{}{2021-01-28}{Include new test cycle for LDM-556 requirements}{Leanne Guy}
\addtohist{}{2022-06-07}{Test campaign LVV-P77 completed and results approved. DM-27337}
}

\setDocCurator{Jeff Carlin}
\setDocUpstreamLocation{\url{https://github.com/lsst-dm/\lsstDocType-\lsstDocNum}}
\setDocUpstreamVersion{\vcsrevision}



\setDocAbstract{
This is the test plan and report for
\textbf{ Gen 3 Butler Acceptance Testing} (LDM-GEN3),
an LSST milestone pertaining to the Data Management Subsystem.\\
This document is based on content automatically extracted from the Jira test database on \docDate.
The most recent change to the document repository was on \vcsDate.
}


\maketitle

\section{Introduction}
\label{sect:intro}


\subsection{Objectives}
\label{sect:objectives}

 The goal of this test is to demonstrate that the Gen3 Butler software
project has sufficiently matured that subsequent DM development should
begin focusing on adoption of Gen3 Butler software are repositories
throughout the DM software project (i.e. that deprecation of Gen2 Butler
usage within the project can begin).



\subsection{System Overview}
\label{sect:systemoverview}

 The Gen3 refactoring of the Butler is central to evolution of the
overall DM software design and has repercussions throughout the rest of
the DM project. ~ This test plan is designed to verify that minimal
requirements have been met and the DM project can now begin the process
of integrating the Gen3 Butler within the pipelines and analysis tools.
~ Those minimal requirements are that:

\begin{enumerate}
\tightlist
\item
  possible to ingest raw dataset types central to the Rubin operations
  and the ongoing development of the data management systems..
\item
  cp\_pipe equivalent under Gen3 is available
\item
  developers can run a pipeline with a single-node using pipetask
\item
  processing supporting development is possible in a reasonable time
  (e.g. a 3-tract RC2 test run can be accomplished within a reasonable
  time)~
\item
  Calibration Product Pipelines (CPP) can be run to support above
  investigations
\item
  Batch Processing System (BPS) is available to support testing at
  larger scales
\end{enumerate}

In addition, at the time these tests occur the Gen3 Butler schema be
considered stable enough that changes no longer occur on a weekly basis
(i.e forced re-ingestion/migration of existing repositories are no
longer a weekly occurrence). Changes requiring wholesale
reingestion/migration may still be required but will occur in a
regimented manner and the choice to allow schema changes without an
accompanying means to migrate old repositories would become a
change-control board (CCB) level
issue.\\[2\baselineskip]\textbf{Applicable Documents:}\\
\citeds{LDM-592}: Data Access Use Cases\\
\citeds{LDM-556}: Data Management Middleware Requirements\\
\citeds{LDM-639}: Data Management Acceptance Test Specification\\[2\baselineskip]


\subsection{Document Overview}
\label{sect:docoverview}

This document was generated from Jira, obtaining the relevant information from the
\href{https://jira.lsstcorp.org/secure/Tests.jspa\#/testPlan/LVV-P77}{LVV-P77}
~Jira Test Plan and related Test Cycles (
\href{https://jira.lsstcorp.org/secure/Tests.jspa\#/testCycle/LVV-C160}{LVV-C160}
\href{https://jira.lsstcorp.org/secure/Tests.jspa\#/testCycle/LVV-C162}{LVV-C162}
).

Section \ref{sect:intro} provides an overview of the test campaign, the system under test (\product{}),
the applicable documentation, and explains how this document is organized.
Section \ref{sect:testplan} provides additional information about the test plan, like for example the configuration
used for this test or related documentation.
Section \ref{sect:personnel} describes the necessary roles and lists the individuals assigned to them.

Section \ref{sect:overview} provides a summary of the test results, including an overview in Table \ref{table:summary},
an overall assessment statement and suggestions for possible improvements.
Section \ref{sect:detailedtestresults} provides detailed results for each step in each test case.

The current status of test plan \href{https://jira.lsstcorp.org/secure/Tests.jspa\#/testPlan/LVV-P77}{LVV-P77} in Jira is \textbf{ Draft }.

\subsection{References}
\label{sect:references}
\renewcommand{\refname}{}
\bibliography{lsst,refs,books,refs_ads,local}


\newpage
\section{Test Plan Details}
\label{sect:testplan}


\subsection{Data Collection}

  Observing is not required for this test campaign.

\subsection{Verification Environment}
\label{sect:hwconf}
  These tests assume a stable weekly stack which supports Gen3 running of
the above, that services that automatically ingest new data can support
on-going ingestion to Gen3 repositories (i.e. DBB shared spaces and OODS
support serving data through Gen3), and that batch processing services
can support pipeline execution of Gen3 products.




\subsection{Related Documentation}


No additional documentation provided.


\subsection{PMCS Activity}

Primavera milestones related to the test campaign:
\begin{itemize}
\item LDM-GEN3
\end{itemize}


\newpage
\section{Personnel}
\label{sect:personnel}

The personnel involved in the test campaign is shown in the following table.

{\small
\begin{longtable}{p{3cm}p{3cm}p{3cm}p{6cm}}
\hline
\multicolumn{2}{r}{T. Plan \href{https://jira.lsstcorp.org/secure/Tests.jspa\#/testPlan/LVV-P77}{LVV-P77} owner:} &
\multicolumn{2}{l}{\textbf{ Robert Gruendl } }\\\hline
\multicolumn{2}{r}{T. Cycle \href{https://jira.lsstcorp.org/secure/Tests.jspa\#/testCycle/LVV-C160}{LVV-C160} owner:} &
\multicolumn{2}{l}{\textbf{
Robert Gruendl }
} \\\hline
\textbf{Test Cases} & \textbf{Assigned to} & \textbf{Executed by} & \textbf{Additional Test Personnel} \\ \hline
\href{https://jira.lsstcorp.org/secure/Tests.jspa#/testCase/LVV-T2264}{LVV-T2264}
& {\small Leanne Guy } & {\small  } &
\begin{minipage}[]{6cm}
\smallskip
{\small  }
\medskip
\end{minipage}
\\ \hline
\href{https://jira.lsstcorp.org/secure/Tests.jspa#/testCase/LVV-T1984}{LVV-T1984}
& {\small Leanne Guy } & {\small  } &
\begin{minipage}[]{6cm}
\smallskip
{\small  }
\medskip
\end{minipage}
\\ \hline
\href{https://jira.lsstcorp.org/secure/Tests.jspa#/testCase/LVV-T1982}{LVV-T1982}
& {\small Leanne Guy } & {\small  } &
\begin{minipage}[]{6cm}
\smallskip
{\small  }
\medskip
\end{minipage}
\\ \hline
\href{https://jira.lsstcorp.org/secure/Tests.jspa#/testCase/LVV-T1987}{LVV-T1987}
& {\small Leanne Guy } & {\small  } &
\begin{minipage}[]{6cm}
\smallskip
{\small  }
\medskip
\end{minipage}
\\ \hline
\href{https://jira.lsstcorp.org/secure/Tests.jspa#/testCase/LVV-T1983}{LVV-T1983}
& {\small Leanne Guy } & {\small  } &
\begin{minipage}[]{6cm}
\smallskip
{\small  }
\medskip
\end{minipage}
\\ \hline
\multicolumn{2}{r}{T. Cycle \href{https://jira.lsstcorp.org/secure/Tests.jspa\#/testCycle/LVV-C162}{LVV-C162} owner:} &
\multicolumn{2}{l}{\textbf{
Undefined }
} \\\hline
\textbf{Test Cases} & \textbf{Assigned to} & \textbf{Executed by} & \textbf{Additional Test Personnel} \\ \hline
\href{https://jira.lsstcorp.org/secure/Tests.jspa#/testCase/LVV-T1985}{LVV-T1985}
& {\small Leanne Guy } & {\small  } &
\begin{minipage}[]{6cm}
\smallskip
{\small  }
\medskip
\end{minipage}
\\ \hline
\end{longtable}
}

\newpage

\section{Test Campaign Overview}
\label{sect:overview}

\subsection{Summary}
\label{sect:summarytable}

{\small
\begin{longtable}{p{2cm}cp{2.3cm}p{8.6cm}p{2.3cm}}
\toprule
\multicolumn{2}{r}{ T. Plan \href{https://jira.lsstcorp.org/secure/Tests.jspa\#/testPlan/LVV-P77}{LVV-P77}:} &
\multicolumn{2}{p{10.9cm}}{\textbf{ LDM-GEN3: Gen 3 Butler Acceptance Testing }} & Draft \\\hline
\multicolumn{2}{r}{ T. Cycle \href{https://jira.lsstcorp.org/secure/Tests.jspa\#/testCycle/LVV-C160}{LVV-C160}:} &
\multicolumn{2}{p{10.9cm}}{\textbf{ LDM-503-GEN3: Gen 3 Butler Acceptance Testing }} & Not Executed \\\hline
\textbf{Test Cases} &  \textbf{Ver.} & \textbf{Status} & \textbf{Comment} & \textbf{Issues} \\\toprule
\href{https://jira.lsstcorp.org/secure/Tests.jspa#/testCase/LVV-T2264}{LVV-T2264}
&  1
& Not Executed &
\begin{minipage}[]{9cm}
\smallskip

\medskip
\end{minipage}
&   \\\hline
\href{https://jira.lsstcorp.org/secure/Tests.jspa#/testCase/LVV-T1984}{LVV-T1984}
&  1
& Not Executed &
\begin{minipage}[]{9cm}
\smallskip

\medskip
\end{minipage}
&   \\\hline
\href{https://jira.lsstcorp.org/secure/Tests.jspa#/testCase/LVV-T1982}{LVV-T1982}
&  1
& Not Executed &
\begin{minipage}[]{9cm}
\smallskip

\medskip
\end{minipage}
&   \\\hline
\href{https://jira.lsstcorp.org/secure/Tests.jspa#/testCase/LVV-T1987}{LVV-T1987}
&  1
& Not Executed &
\begin{minipage}[]{9cm}
\smallskip

\medskip
\end{minipage}
&   \\\hline
\href{https://jira.lsstcorp.org/secure/Tests.jspa#/testCase/LVV-T1983}{LVV-T1983}
&  1
& Not Executed &
\begin{minipage}[]{9cm}
\smallskip

\medskip
\end{minipage}
&   \\\hline
\multicolumn{2}{r}{ T. Cycle \href{https://jira.lsstcorp.org/secure/Tests.jspa\#/testCycle/LVV-C162}{LVV-C162}:} &
\multicolumn{2}{p{10.9cm}}{\textbf{ LDM-503-GEN3: Gen 3 Ingest raw dataset }} & Not Executed \\\hline
\textbf{Test Cases} &  \textbf{Ver.} & \textbf{Status} & \textbf{Comment} & \textbf{Issues} \\\toprule
\href{https://jira.lsstcorp.org/secure/Tests.jspa#/testCase/LVV-T1985}{LVV-T1985}
&  1
& Not Executed &
\begin{minipage}[]{9cm}
\smallskip

\medskip
\end{minipage}
&   \\\hline
\caption{Test Campaign Summary}
\label{table:summary}
\end{longtable}
}

\subsection{Overall Assessment}
\label{sect:overallassessment}

Not yet available.

\subsection{Recommended Improvements}
\label{sect:recommendations}

Not yet available.

\newpage
\section{Detailed Test Results}
\label{sect:detailedtestresults}

\subsection{Test Cycle LVV-C160 }

Open test cycle {\it \href{https://jira.lsstcorp.org/secure/Tests.jspa#/testrun/LVV-C160}{LDM-503-GEN3: Gen 3 Butler Acceptance Testing}} in Jira.

Test Cycle name: LDM-503-GEN3: Gen 3 Butler Acceptance Testing\\
Status: Not Executed

This test cycle is meant to demonstrate that the Gen3 butler and
associated database and pipeline interfaces have matured to the point
where they can replace the Gen2 butler. ~The test cases outlined here:

\begin{enumerate}
\tightlist
\item
  use a series of modest pipeline executions to show that the Gen3
  software can support all future pipeline development, ~
\item
  those pipeline executions also show that a batch processing system
  (BPS) is available to enable that processing, and
\item
  demonstrate through inspection that documentation for developers
  exists
\item
  confirm that pipeline developers do not know of blockers if all future
  development assumes Gen3 Butler.
\end{enumerate}

\subsubsection{Software Version/Baseline}
Not provided.

\subsubsection{Configuration}
Gen3 Butler repositories with test data are available within DBB spaces.
~Weekly DM stack has Gen3 and BPS elements present for tests.

\subsubsection{Test Cases in LVV-C160 Test Cycle}

\paragraph{ LVV-T2264 - Butler Gen3 maturity sufficient to support future pipeline development. }\mbox{}\\

Version \textbf{1}.
Open  \href{https://jira.lsstcorp.org/secure/Tests.jspa#/testCase/LVV-T2264}{\textit{ LVV-T2264 } }
test case in Jira.

This test is meant to verify that Butler Gen3 maturity is sufficient to
provide comparable (or better) pipeline capabilities and results to
those available under Butler Gen2.

\textbf{ Preconditions}:\\


Execution status: {\bf Not Executed }

Final comment:\\


Detailed steps results:

\begin{tabular}{p{2cm}p{14cm}}
\toprule
Step 1 & Step Execution Status: \textbf{ Not Executed } \\ \hline
\end{tabular}
 Description \\
{\footnotesize

}
\hdashrule[0.5ex]{\textwidth}{1pt}{3mm}
  Expected Result \\
{\footnotesize

}
\hdashrule[0.5ex]{\textwidth}{1pt}{3mm}
  Actual Result \\
{\footnotesize

}

\paragraph{ LVV-T1984 - Demonstrate documentation/examples of Gen3 usage and cp\_pipe
equivalent. }\mbox{}\\

Version \textbf{1}.
Open  \href{https://jira.lsstcorp.org/secure/Tests.jspa#/testCase/LVV-T1984}{\textit{ LVV-T1984 } }
test case in Jira.

Demonstrate the existence of fundamental documentation necessary to aid
Gen2 users with the transition to Gen3 use.

\textbf{ Preconditions}:\\


Execution status: {\bf Not Executed }

Final comment:\\


Detailed steps results:

\begin{tabular}{p{2cm}p{14cm}}
\toprule
Step 1 & Step Execution Status: \textbf{ Not Executed } \\ \hline
\end{tabular}
 Description \\
{\footnotesize
Identify document(s), web-pages, archived presentations, or example
notebooks that provide documentation and/or examples of Gen3
functionality.

}
\hdashrule[0.5ex]{\textwidth}{1pt}{3mm}
  Expected Result \\
{\footnotesize
Document reference(s) or URL(s) for such documentation.

}
\hdashrule[0.5ex]{\textwidth}{1pt}{3mm}
  Actual Result \\
{\footnotesize

}

\paragraph{ LVV-T1982 - Run a pipeline on a single node using pipetask. }\mbox{}\\

Version \textbf{1}.
Open  \href{https://jira.lsstcorp.org/secure/Tests.jspa#/testCase/LVV-T1982}{\textit{ LVV-T1982 } }
test case in Jira.

To show that individual users have the ability to run either locally (w/
sqlite) or generally (w/ Postgres) using Gen3 Butler infrastructure.

\textbf{ Preconditions}:\\
This test requires that Gen3 Butler infrastructure and underlying pipets
have been integrated. ~It further requires (in spirit) that gen3 schema
stability has been reached to facilitate comparison of pipeline results
with further stack development can be compared.

Execution status: {\bf Not Executed }

Final comment:\\


Detailed steps results:

\begin{tabular}{p{2cm}p{14cm}}
\toprule
Step 1 & Step Execution Status: \textbf{ Not Executed } \\ \hline
\end{tabular}
 Description \\
{\footnotesize
Setup stack, identify inputs, pipetask execution of standard ci\_hsc
run.

}
\hdashrule[0.5ex]{\textwidth}{1pt}{3mm}
  Test Data \\
 {\footnotesize
ci\_hsc raw repository within a Gen3 Butler repo

}
\hdashrule[0.5ex]{\textwidth}{1pt}{3mm}
  Expected Result \\
{\footnotesize
Pipeline executes standard reduction without failure.

}
\hdashrule[0.5ex]{\textwidth}{1pt}{3mm}
  Actual Result \\
{\footnotesize

}

\paragraph{ LVV-T1987 - Run Calibration Products Processing (CPP) }\mbox{}\\

Version \textbf{1}.
Open  \href{https://jira.lsstcorp.org/secure/Tests.jspa#/testCase/LVV-T1987}{\textit{ LVV-T1987 } }
test case in Jira.

Demonstrate that basic calibration processing from Gen2 era has been
enabled within Gen3 environment. ~ This test is not concerned with large
scales but merely demonstrates that Gen3 capability to generate
calibration products (i.e. they are no longer required to be generated
in Gen2 and then migrated to Gen3).

\textbf{ Preconditions}:\\


Execution status: {\bf Not Executed }

Final comment:\\


Detailed steps results:

\begin{tabular}{p{2cm}p{14cm}}
\toprule
Step 1 & Step Execution Status: \textbf{ Not Executed } \\ \hline
\end{tabular}
 Description \\
{\footnotesize
Identify an existing or instantiate a new Gen3 repository with raw bias,
dark, and flat observations.

}
\hdashrule[0.5ex]{\textwidth}{1pt}{3mm}
  Test Data \\
 {\footnotesize
It is preferred these data be early observatory products (i.e. either
AuxTel/LATISS or ComCam).

}
\hdashrule[0.5ex]{\textwidth}{1pt}{3mm}
  Expected Result \\
{\footnotesize
A Gen3 repo with appropriate raw data products.

}
\hdashrule[0.5ex]{\textwidth}{1pt}{3mm}
  Actual Result \\
{\footnotesize

}
\begin{tabular}{p{2cm}p{14cm}}
\toprule
Step 2 & Step Execution Status: \textbf{ Not Executed } \\ \hline
\end{tabular}
 Description \\
{\footnotesize
Create master bias, dark and flat products from the raw products.

}
\hdashrule[0.5ex]{\textwidth}{1pt}{3mm}
  Expected Result \\
{\footnotesize
A master bias, dark, and flat calibration product.\\[2\baselineskip]

}
\hdashrule[0.5ex]{\textwidth}{1pt}{3mm}
  Actual Result \\
{\footnotesize

}

\paragraph{ LVV-T1983 - Mini RC2 processing capability }\mbox{}\\

Version \textbf{1}.
Open  \href{https://jira.lsstcorp.org/secure/Tests.jspa#/testCase/LVV-T1983}{\textit{ LVV-T1983 } }
test case in Jira.

Demonstrate that a typical 3-tract RC2 data processing is possible using
the Gen3 system and the nascent Batch Production Service (BPS). ~This
test is meant to demonstrate that Gen3 + BPS systems are capable of
supporting future DM development by demonstrating that processing
routinely used by developers for benchmarking/testing improvements can
be performed in a reasonable time. ~

\textbf{ Preconditions}:\\


Execution status: {\bf Not Executed }

Final comment:\\


Detailed steps results:

\begin{tabular}{p{2cm}p{14cm}}
\toprule
Step 1 & Step Execution Status: \textbf{ Not Executed } \\ \hline
\end{tabular}
 Description \\
{\footnotesize
setup environment\\
identify input data products

}
\hdashrule[0.5ex]{\textwidth}{1pt}{3mm}
  Test Data \\
 {\footnotesize
RC2 raw repo

}
\hdashrule[0.5ex]{\textwidth}{1pt}{3mm}
  Example Code \\
{\footnotesize
\# for example on lsstdev-* resources at NCSA\\[2\baselineskip]export
lsstsw\_root=/software/lsstsw/stack\\
export EUPS\_TAG=``w\_2020\_46''\\[2\baselineskip]source
/opt/rh/devtoolset-8/enable\\
source \$\{lsstsw\_root\}/loadLSST.bash\\
setup lsst\_distrib -t \$\{EUPSTAG\}\\[2\baselineskip]

}
\hdashrule[0.5ex]{\textwidth}{1pt}{3mm}
  Expected Result \\
{\footnotesize
software environment ready for job submission\\[2\baselineskip]

}
\hdashrule[0.5ex]{\textwidth}{1pt}{3mm}
  Actual Result \\
{\footnotesize

}
\begin{tabular}{p{2cm}p{14cm}}
\toprule
Step 2 & Step Execution Status: \textbf{ Not Executed } \\ \hline
\end{tabular}
 Description \\
{\footnotesize
BPS pipeline submission

}
\hdashrule[0.5ex]{\textwidth}{1pt}{3mm}
  Example Code \\
{\footnotesize
pipetask qgraph -d ``tract = 9615 and instrument='HSC' and
skymap='hsc\_rings\_v1''' \textbackslash{}\\
-b \{gen3\_repo\}/\{version\}/butler.yaml \textbackslash{}\\
-i HSC/calib,HSC/raw/all,HSC/masks,refcats,skymaps \textbackslash{}\\
-p /home/madamow/gen2-to-gen3/bps/HSC-RC2.yaml \textbackslash{}\\
-q
/home/madamow/gen2-to-gen3/bps/submit/RC2/w\_2020\_42/DM-27244/20201102T10h22m03s/RC2\_w\_2020\_42\_DM-27244\_20201102T10h22m03s.pickle

}
\hdashrule[0.5ex]{\textwidth}{1pt}{3mm}
  Expected Result \\
{\footnotesize
Pipeline execution is successful. ~An estimate of ~the compute resources
used (\# cores, memory, wall time) for each execution should be
reported.

}
\hdashrule[0.5ex]{\textwidth}{1pt}{3mm}
  Actual Result \\
{\footnotesize

}


\subsection{Test Cycle LVV-C162 }

Open test cycle {\it \href{https://jira.lsstcorp.org/secure/Tests.jspa#/testrun/LVV-C162}{LDM-503-GEN3: Gen 3 Ingest raw dataset}} in Jira.

Test Cycle name: LDM-503-GEN3: Gen 3 Ingest raw dataset\\
Status: Not Executed

In the context of the milestone LDM-503-GEN3, Gen 3 Butler readiness,
this test cycle is defining the configuration and the dataset for
running a generic \textbf{Raw Data Ingestion Into Gen3 Butler} test
case. ~ There are currently 5 data sources that require verification as
they are the central products that will be produced by Rubin or are used
as precursor sets in the development/verification of the data management
software systems. ~The current raw data products that are deemed central
to DM development and testing are those from AuxTel/LATISS, ComCam, and
~precursor data from HyperSuprimeCam (HSC). ~Note, further tests using
LSSTCam (currently only preliminary BOT data from the SLAC test stand
are available) or precursor sets from the Dark Energy Camera (DECam)
could be added but since these types do not exactly fit the central
model used for LSST they are not tied directly to requirements.

\subsubsection{Software Version/Baseline}
LSST DM Stack with Gen3 Butler.

\subsubsection{Configuration}
Three separate raw data types, those from: AuxTel/LATISS, ComCam, and
HSC (e.g. a CI\_HSC raw) should be ingested when this test is executed.

\subsubsection{Test Cases in LVV-C162 Test Cycle}

\paragraph{ LVV-T1985 - Verify daf\_butler raw data ingest }\mbox{}\\

Version \textbf{1}.
Open  \href{https://jira.lsstcorp.org/secure/Tests.jspa#/testCase/LVV-T1985}{\textit{ LVV-T1985 } }
test case in Jira.

Demonstrate that a raw data type can be successfully ingested into a
Butler repository. ~

\textbf{ Preconditions}:\\
In order to run this test, a Gen3 daf butler should be deployed and
ready to use, with access to the filesystems where the raw data to
ingest is stored.

Execution status: {\bf Not Executed }

Final comment:\\


Detailed steps results:

\begin{tabular}{p{2cm}p{14cm}}
\toprule
Step 1 & Step Execution Status: \textbf{ Not Executed } \\ \hline
\end{tabular}
 Description \\
{\footnotesize
Identify data for ingestion~{HSC RC2}⁠ and make ~a copy at a location
for the test. ~While a suggestion is provided in
{/project/shared/hsc/COSMOS/2014-03-27/}⁠ for a location where such data
can be found, the actual data used can be left to the discretion of the
person(s) executing the test with the added suggestion that relatively
recent data are more likely to reflect the current observatory system
state.\\[2\baselineskip]

}
\hdashrule[0.5ex]{\textwidth}{1pt}{3mm}
  Test Data \\
 {\footnotesize
{/project/shared/hsc/COSMOS/2014-03-27/}⁠~⁠~

}
\hdashrule[0.5ex]{\textwidth}{1pt}{3mm}
  Expected Result \\
{\footnotesize
One or more raw data sets are identified and made available.

}
\hdashrule[0.5ex]{\textwidth}{1pt}{3mm}
  Actual Result \\
{\footnotesize

}
\begin{tabular}{p{2cm}p{14cm}}
\toprule
Step 2 & Step Execution Status: \textbf{ Not Executed } \\ \hline
\end{tabular}
 Description \\
{\footnotesize
Identify data for ingestion~{AuxTel/LATISS}⁠ and make ~a copy at a
location for the test. ~While a suggestion is provided in
{/project/shared/auxTel/\_parent/raw/2021-03-23/}⁠ for a location where
such data can be found, the actual data used can be left to the
discretion of the person(s) executing the test with the added suggestion
that relatively recent data are more likely to reflect the current
observatory system state.\\[2\baselineskip]

}
\hdashrule[0.5ex]{\textwidth}{1pt}{3mm}
  Test Data \\
 {\footnotesize
{/project/shared/auxTel/\_parent/raw/2021-03-23/}⁠~⁠~

}
\hdashrule[0.5ex]{\textwidth}{1pt}{3mm}
  Expected Result \\
{\footnotesize
One or more raw data sets are identified and made available.

}
\hdashrule[0.5ex]{\textwidth}{1pt}{3mm}
  Actual Result \\
{\footnotesize

}
\begin{tabular}{p{2cm}p{14cm}}
\toprule
Step 3 & Step Execution Status: \textbf{ Not Executed } \\ \hline
\end{tabular}
 Description \\
{\footnotesize
Identify data for ingestion~{ComCam}⁠ and make ~a copy at a location for
the test. ~While a suggestion is provided in
{/project/shared/comCam/\_parent/raw/2021-05-14/2021051400003/}⁠ for a
location where such data can be found, the actual data used can be left
to the discretion of the person(s) executing the test with the added
suggestion that relatively recent data are more likely to reflect the
current observatory system state.\\[2\baselineskip]

}
\hdashrule[0.5ex]{\textwidth}{1pt}{3mm}
  Test Data \\
 {\footnotesize
{/project/shared/comCam/\_parent/raw/2021-05-14/2021051400003/}⁠~⁠~

}
\hdashrule[0.5ex]{\textwidth}{1pt}{3mm}
  Expected Result \\
{\footnotesize
One or more raw data sets are identified and made available.

}
\hdashrule[0.5ex]{\textwidth}{1pt}{3mm}
  Actual Result \\
{\footnotesize

}
\begin{tabular}{p{2cm}p{14cm}}
\toprule
Step 4 & Step Execution Status: \textbf{ Not Executed } \\ \hline
\end{tabular}
 Description \\
{\footnotesize
Verify that Butler repository is available for the {HSC RC2}⁠ (Note this
needs to be a test repository rather than the central repository as the
raw data should not already be present in the repository for this test.)

}
\hdashrule[0.5ex]{\textwidth}{1pt}{3mm}
  Test Data \\
 {\footnotesize
{url 1}⁠~

}
\hdashrule[0.5ex]{\textwidth}{1pt}{3mm}
  Example Code \\
{\footnotesize
\# create empty Gen3 repo (for ComCam data)\\[2\baselineskip]butler
create repo\\
butler register-instrument repo lsst.obs.lsst.LsstComCam

}
\hdashrule[0.5ex]{\textwidth}{1pt}{3mm}
  Expected Result \\
{\footnotesize

}
\hdashrule[0.5ex]{\textwidth}{1pt}{3mm}
  Actual Result \\
{\footnotesize

}
\begin{tabular}{p{2cm}p{14cm}}
\toprule
Step 5 & Step Execution Status: \textbf{ Not Executed } \\ \hline
\end{tabular}
 Description \\
{\footnotesize
Verify that Butler repository is available for the {AuxTel/LATISS}⁠
(Note this needs to be a test repository rather than the central
repository as the raw data should not already be present in the
repository for this test.)

}
\hdashrule[0.5ex]{\textwidth}{1pt}{3mm}
  Test Data \\
 {\footnotesize
{url 2}⁠~

}
\hdashrule[0.5ex]{\textwidth}{1pt}{3mm}
  Example Code \\
{\footnotesize
\# create empty Gen3 repo (for ComCam data)\\[2\baselineskip]butler
create repo\\
butler register-instrument repo lsst.obs.lsst.LsstComCam

}
\hdashrule[0.5ex]{\textwidth}{1pt}{3mm}
  Expected Result \\
{\footnotesize

}
\hdashrule[0.5ex]{\textwidth}{1pt}{3mm}
  Actual Result \\
{\footnotesize

}
\begin{tabular}{p{2cm}p{14cm}}
\toprule
Step 6 & Step Execution Status: \textbf{ Not Executed } \\ \hline
\end{tabular}
 Description \\
{\footnotesize
Verify that Butler repository is available for the {ComCam}⁠ (Note this
needs to be a test repository rather than the central repository as the
raw data should not already be present in the repository for this test.)

}
\hdashrule[0.5ex]{\textwidth}{1pt}{3mm}
  Test Data \\
 {\footnotesize
{url 3}⁠~

}
\hdashrule[0.5ex]{\textwidth}{1pt}{3mm}
  Example Code \\
{\footnotesize
\# create empty Gen3 repo (for ComCam data)\\[2\baselineskip]butler
create repo\\
butler register-instrument repo lsst.obs.lsst.LsstComCam

}
\hdashrule[0.5ex]{\textwidth}{1pt}{3mm}
  Expected Result \\
{\footnotesize

}
\hdashrule[0.5ex]{\textwidth}{1pt}{3mm}
  Actual Result \\
{\footnotesize

}
\begin{tabular}{p{2cm}p{14cm}}
\toprule
Step 7 & Step Execution Status: \textbf{ Not Executed } \\ \hline
\end{tabular}
 Description \\
{\footnotesize
Ingest {HSC RC2}⁠~raw data into repo

}
\hdashrule[0.5ex]{\textwidth}{1pt}{3mm}
  Test Data \\
 {\footnotesize
{url 1}⁠~

}
\hdashrule[0.5ex]{\textwidth}{1pt}{3mm}
  Example Code \\
{\footnotesize
butler ingest-raws repo raw

}
\hdashrule[0.5ex]{\textwidth}{1pt}{3mm}
  Expected Result \\
{\footnotesize
Tool reports data ingest successful for {HSC RC2}⁠ into {url 1}⁠~

}
\hdashrule[0.5ex]{\textwidth}{1pt}{3mm}
  Actual Result \\
{\footnotesize

}
\begin{tabular}{p{2cm}p{14cm}}
\toprule
Step 8 & Step Execution Status: \textbf{ Not Executed } \\ \hline
\end{tabular}
 Description \\
{\footnotesize
Ingest {AuxTel/LATISS}⁠~raw data into repo

}
\hdashrule[0.5ex]{\textwidth}{1pt}{3mm}
  Test Data \\
 {\footnotesize
{url 2}⁠~

}
\hdashrule[0.5ex]{\textwidth}{1pt}{3mm}
  Example Code \\
{\footnotesize
butler ingest-raws repo raw

}
\hdashrule[0.5ex]{\textwidth}{1pt}{3mm}
  Expected Result \\
{\footnotesize
Tool reports data ingest successful for {AuxTel/LATISS}⁠ into {url 2}⁠~

}
\hdashrule[0.5ex]{\textwidth}{1pt}{3mm}
  Actual Result \\
{\footnotesize

}
\begin{tabular}{p{2cm}p{14cm}}
\toprule
Step 9 & Step Execution Status: \textbf{ Not Executed } \\ \hline
\end{tabular}
 Description \\
{\footnotesize
Ingest {ComCam}⁠~raw data into repo

}
\hdashrule[0.5ex]{\textwidth}{1pt}{3mm}
  Test Data \\
 {\footnotesize
{url 3}⁠~

}
\hdashrule[0.5ex]{\textwidth}{1pt}{3mm}
  Example Code \\
{\footnotesize
butler ingest-raws repo raw

}
\hdashrule[0.5ex]{\textwidth}{1pt}{3mm}
  Expected Result \\
{\footnotesize
Tool reports data ingest successful for {ComCam}⁠ into {url 3}⁠~

}
\hdashrule[0.5ex]{\textwidth}{1pt}{3mm}
  Actual Result \\
{\footnotesize

}
\begin{tabular}{p{2cm}p{14cm}}
\toprule
Step 10 & Step Execution Status: \textbf{ Not Executed } \\ \hline
\end{tabular}
 Description \\
{\footnotesize
Query repository to verify that ingestion of {HSC RC2}⁠~ occurred.

}
\hdashrule[0.5ex]{\textwidth}{1pt}{3mm}
  Test Data \\
 {\footnotesize
{url 1}⁠~

}
\hdashrule[0.5ex]{\textwidth}{1pt}{3mm}
  Expected Result \\
{\footnotesize
{HSC RC2}⁠ data are found by query.~

}
\hdashrule[0.5ex]{\textwidth}{1pt}{3mm}
  Actual Result \\
{\footnotesize

}
\begin{tabular}{p{2cm}p{14cm}}
\toprule
Step 11 & Step Execution Status: \textbf{ Not Executed } \\ \hline
\end{tabular}
 Description \\
{\footnotesize
Query repository to verify that ingestion of {AuxTel/LATISS}⁠~ occurred.

}
\hdashrule[0.5ex]{\textwidth}{1pt}{3mm}
  Test Data \\
 {\footnotesize
{url 2}⁠~

}
\hdashrule[0.5ex]{\textwidth}{1pt}{3mm}
  Expected Result \\
{\footnotesize
{AuxTel/LATISS}⁠ data are found by query.~

}
\hdashrule[0.5ex]{\textwidth}{1pt}{3mm}
  Actual Result \\
{\footnotesize

}
\begin{tabular}{p{2cm}p{14cm}}
\toprule
Step 12 & Step Execution Status: \textbf{ Not Executed } \\ \hline
\end{tabular}
 Description \\
{\footnotesize
Query repository to verify that ingestion of {ComCam}⁠~ occurred.

}
\hdashrule[0.5ex]{\textwidth}{1pt}{3mm}
  Test Data \\
 {\footnotesize
{url 3}⁠~

}
\hdashrule[0.5ex]{\textwidth}{1pt}{3mm}
  Expected Result \\
{\footnotesize
{ComCam}⁠ data are found by query.~

}
\hdashrule[0.5ex]{\textwidth}{1pt}{3mm}
  Actual Result \\
{\footnotesize

}




% This appendix is put in as part of the template. You may edit and add to it.
% It is not overwritten by Docsteady.

\newpage
\appendix
\section{Documentation}
The verification process is defined in \citeds{LSE-160}.
The use of Docsteady to format Jira information in various test and planing documents is
described in \citeds{DMTN-140} and practical commands are given in \citeds{DMTN-178}.

\section{Acronyms used in this document}\label{sec:acronyms}
\input{acronyms.tex}

\newpage

% Uncomment this if Docsteady makes you additional appendix
%% generated from JIRA project LVV
% using template at /Users/womullan/LSSTgit/docsteady/docsteady/templates/tpr-appendix.latex.jinja2.
% using docsteady version 2.2.post4+g100285e.d20210519
% Please do not edit -- update information in Jira instead
\section{Traceability}

\begin{longtable}{p{3cm}p{3cm}L{9cm}}
\hline
\textbf{Test Case} & \textbf{VE Key} & \textbf{VE Summary} \\ \hline
\href{https://jira.lsstcorp.org/secure/Tests.jspa#/testCase/LVV-T1982}{LVV-T1982} &
 & \\ \hline
\href{https://jira.lsstcorp.org/secure/Tests.jspa#/testCase/LVV-T1983}{LVV-T1983} &
 & \\ \hline
\href{https://jira.lsstcorp.org/secure/Tests.jspa#/testCase/LVV-T1984}{LVV-T1984} &
 & \\ \hline
\href{https://jira.lsstcorp.org/secure/Tests.jspa#/testCase/LVV-T1985}{LVV-T1985} &
  \href{https://jira.lsstcorp.org/browse/LVV-130}{LVV-130}
  & DMS-REQ-0299-V-01: Data Product Ingest
 \\ \cdashline{2-3}
\hline
\href{https://jira.lsstcorp.org/secure/Tests.jspa#/testCase/LVV-T1987}{LVV-T1987} &
  \href{https://jira.lsstcorp.org/browse/LVV-120}{LVV-120}
  & DMS-REQ-0289-V-01: Calibration Production Processing
 \\ \cdashline{2-3}
\hline
\end{longtable}


\end{document}
